\section{Introduction}
The \textit{n}-queens problem is a puzzle where the solution is to place \textit{n} queens on a \textit{n}x\textit{n} board, such that no two queens threaten each other. This means that no two queens can be in the same row, column and diagonals.
\section{QueensLogic}
The two classes \texttt{QueensGUI} and \texttt{ShowBoard} are only relevant to drawing the solution and we will therefore only describe the methods in the class \texttt{QueensLogic}. \texttt{QueensLogic} is the class responsible for all logic relevant to solve the n-queens problem and has the following methods:
\begin{description}
	\item[InitializeGame] Sets $N$, the width and height of the game board and create a 2D int array to represent the chess board. Lastly it calls \textit{createBDD} and \textit{setInvalids}.
	\item[getGameBoard] Return the 2D int array representing the chessboard.
	\item[createBDD] Uses the javabdd package to initialize a BDD with 2.000.000 nodes and 200.000 cache as suggested. It then creates a BDD with N*N variables -- one for each chessboard position. Lastly it calls \texttt{createRules}.
	\item[createRules] Creates the first of the two rules for the BDD to solve the n-queens problem. The first rule is that there must be a queen in any column. It calls \texttt{noCaptureRule} to create the second rule.
	\item[noCaptureRule] Creates the second rule, which specifies that no queen must be able to capture another.
	\item[position] Converts a board position of the form $column, row$ to variable number in the BDD between 0-63.	
	\item[isInvalid] Checks if setting a variable in the BDD to true makes the BDD false. 
	\item[setInvalids] Goes through all chessboard positions and calls isInvalid to check if placing a queen on a particular tile makes the BDD false.
	\item[setRemainingVailds] Set all positions still valid on the chessboard to 1 which makes the GUI place a queen there.
	\item[insertQueen] Returns without doing anything if the users is trying to place a queen in an invalid position or in a position already containing a queen. If that is not the case, it places a queen on the given position, updates the BDD and the invalid positions by calling \textit{setInvalids}. Lastly, if there is only one remaining solution to the BDD, it calls \texttt{setRemainingValids}.
\end{description}

\section{Rule implementation}
We will go through how the two rules, at-least-one-queen-per-row and no-queen-can-capture-another are implemented. 
The first rule one-queen-per-row is implemented by:
\begin{enumerate}
 \item Creating a sub BDD that is false for every column.
 \item Going through every position on the chessboard and checking whether or not it is possible to place a queen there.
\end{enumerate}

The second rule no-queen-can-capture-another is implemented by going through all positions in the chessboard and:
\begin{enumerate}
	\item Create a sub BDD representing the that no queen can capture another. The BDD is initialized to true.
 	\item Go through all column positions with the given row position and checking that there is no queen there.
 	\item Go through all row positions with the given column position and checking that there is no queen there.
 	\item Go through all the diagonals positions with the given column and row position and checking that there is no queen there.
\end{enumerate}

\section{Conclusion}
The implemented logic appears to block the invalid cells correctly resulting in a correct solution. The auto complete when only one solution remains also seems to function corretly.