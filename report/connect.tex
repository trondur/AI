\section{Connect 4}

\subsection{Minimax}
\textit{MidOrFeed} uses a minimax algorithm to calculate the optimal move. A minimax algorithm calculates the best move based on the assumption that the adversary has takes his best move previous to our previous move. In plainer terms, the algorithm seeks to maximize his score based the knowledge that the adversary will try to minimize our gain (maximize his gain).

\subsection{Cutoff function}
When we are using a minimax algorithm for Connect Four, it search through all the possible game states reachable. In Connect Four the searchspace is so large that it is unfeasible to examine all possible move and their consequences until only a few moves are left. Therefore it is necessary to restrict the search depth with a cut-off function. \textit{MidOrFeed's} cut-off function is implemented using only tree depth. There is a hardcoded maximal search depth, and the minimax algorithm will return with best possible move so far, if it has reached the maximal tree depth.

\subsection{$\alpha$-$\beta$ pruning}
Even with a cut-off function the searchspace is still quite large. To further reduce the searchspace \textit{MidOrFeed's} minimax algorithm implements $\alpha$-$\beta$ pruning. $\alpha$-$\beta$ pruning. $\alpha$-$\beta$ pruning allows \textit{MidOrFeed} to end the exploration of a branch early. Let us consider a n example, where the minimax algorithm is in the maximum phase in a state, where branch $A$ will return a score of 4 and branch $B$ leads to another branch, which has two option which give 3 and 5. In this example our $\alpha$ is 4. If we choose to follow branch $B$, the minimax algorithm will be in the minimum phase. So when the algorithm examine option 3, minimum will always pick that option. So we do not need to examine $B$'s other options, they are pruned, since branch $A$ will always be better (if both players play optimally).

To optimize the $\alpha$-$\beta$ pruning \textit{MidOrFeed} explores the middle column first and then out towards the edges. In most cases this speeds up the pruning, since the best moves, especially at the start of the game, are usually found close to or in the middle column. And when the best moves are found earlier, there is a bigger chance that some branches can be pruned. This makes the searchspace smaller and in turn lets us use a higher maximal search depth. However if there are better moves further from the center column this approach will slow down the pruning.

\subsection{Heuristics}
The heuristic used scores a state based on amount of possible lines of four are started and possible to finish, compared to the opponent. An open line of length three are worth more than any line with two spots filled which is worth more than one. A line is worth the number of spots filled squared. This means a line of two are worth four while a line of three is worth nine. If a line is blocked it is worth zero. If the opponent has a line it is worth the negated amount.

The lines are starting from each field on the row towards the right, the two diagonals in the right direction and up the column. The fields in the three times three square in the upper right corner are not counted, since no lines can start from there. This means fields towards the middle of the board will be used in multiple lines, resulting in a board with an open line in the middle are worth more than in the sides.

This gives an heuristic where more possible lines of four for you increases the value while possible lines for the opponent decreases the value. But lines towards the middle and lines with more places filled are worth more.

\subsection{Pre coded moves}
As a result of the chosen heuristics and the rather naively implemented cut-off function \textit{MidOrFeed} occasionally makes very bad moves at the start of the game, when the searchspace is very large. To prevent the bad moves at the beginning, we compared the starting moves to the actual perfect moves using a connect 4 solver.\footnote{\url{http://connect4.gamesolver.org}}

We identified the worst moves, and hardcoded \textit{MidOrFeed's} actions in certain early game scenarios. This method is used at most until turn 7.

\subsection{Further improvements}
While testing \textit{MidOrFeed}, we have seen that the heuristics we use are not optimal. Even though a state might be worse according to the heuristic, it might be better in the end because it allows a better position in the future. To further improve the heuristic looking at the placement of possible lines of four. A row of three might be better than a column of three. There's also the possibility of looking building towards certain possitions which are a certain win if no mistakes are made.

The cut-off function is also implemented rather naively. A better approach would be to consider if a gamestate was suitable for cut-off rather than just having a fixed tree depth.